%作者:王美庭
%Email:wangmeiting92@gmail.com

%=============导言区通用设置===============

%%%=====主题设置*******
\usetheme{default} %设置整体上的主题
\usefonttheme[onlymath]{serif} %设置数学公式字体为衬线字体
\useoutertheme[subsection=false]{miniframes}
\useinnertheme{circles}
%\usecolortheme{wolverine}
\setbeamertemplate{navigation symbols}{} %移除所有的导航栏
\setbeamertemplate{caption}[numbered] %展示图表计数器
%\setbeamertemplate{footline}[text line]{\hfill\strut\insertframenumber{} / \inserttotalframenumber} %在页面右下角添加“当前帧数 / 总帧数”,\strut表示生成一个宽度为0,总高度等于行距的不可见支柱
\setbeamersize{text margin left=0.8cm, text margin right=0.8cm} %设置文字区域左右留白的宽度
\setbeamertemplate{itemize subitem}{--} %将无序列表的标志样式改为--(第二层)
%\setbeamertemplate{itemize item}[circle] %将无序列表的标志样式改为圆形(第一层)
%\setbeamertemplate{enumerate item}[circle] %将有序列表的标志样式改为圆形(第一层)
%\setbeamertemplate{section in toc}[circle] %设置section in toc为圆形包含数字


%%%=====宏包使用********
\usepackage[space=true,hyperref,UTF8]{ctex} %支持中文,加入超链接宏包,设置UTF8编码
\usepackage{graphicx} %引入图片所需宏包
\graphicspath{{figures/}} %设置图像插入路径
\usepackage{lipsum} %可输入英文假文
\usepackage{ragged2e}
\justifying\let\raggedright\justifying %设置段落对齐方式为两端对齐
\usepackage{float} %其H参数可以让浮动环境不再浮动
\usepackage{array} %提供了更多的表格列说明符,以及修正了一些表格显示上的问题
\usepackage{booktabs} %以使用学术上常见的三线表命令
\usepackage{dcolumn} %可使用小数点对齐列说明符
\usepackage{setspace}
\setstretch{1} %设置行间距的因子(默认为1)
\usepackage{calligra} %手写字体
\usepackage[T1]{fontenc} %以实现更多的字体功能,如默认字体下的斜体加粗
\usepackage{ifthen}


%设置计数器
\newcounter{mycntthm} %设置新的计数器
\newcounter{mycntexam}
\setcounter{mycntthm}{0} %设定计数器默认值
\setcounter{mycntexam}{0}


%开关设置
\newcommand{\thmseriesnamestyle}{Chinese} %Chinese表示设置定理类名称为中文,English表示设置定理类名称为英文
\newcommand{\thmseriesnumbering}{true} %true表示对定理类环境进行编号,false表示不对定理类环境进行编号

\ifthenelse{\equal{\thmseriesnamestyle}{Chinese}}{%
	\newcommand{\dfnname}{定义}
	\newcommand{\lemmaname}{引理}
	\renewcommand{\thmname}{定理}
	\newcommand{\coroname}{推论}
	\renewcommand{\proofname}{证明.}
	\newcommand{\examname}{例}
	\newcommand{\alertname}{注意}
}%
{%
	\newcommand{\dfnname}{Definition}
	\newcommand{\lemmaname}{Lemma}
	\renewcommand{\thmname}{Theorem}
	\newcommand{\coroname}{Corollary}
	\renewcommand{\proofname}{Proof.}
	\newcommand{\examname}{Example}
	\newcommand{\alertname}{Alert}
}

\ifthenelse{\equal{\thmseriesnumbering}{true}}{%
	\newcommand{\mycntthm}{\stepcounter{mycntthm}\themycntthm}
	\newcommand{\mycntexam}{\stepcounter{mycntexam}\themycntexam}
}%
{%
	\newcommand{\mycntthm}{}
	\newcommand{\mycntexam}{}
}


%自定义颜色
\colorlet{temp}{violet} %定义环境颜色
\colorlet{fgmyupdfncolor}{temp}
\colorlet{bgmyupdfncolor}{temp!20}
\colorlet{bgmylowdfncolor}{temp!10}
\setbeamercolor{myupdfncolor}{fg=fgmyupdfncolor,bg=bgmyupdfncolor}
\setbeamercolor{mylowdfncolor}{fg=black,bg=bgmylowdfncolor}

\definecolor{fgmyupthmcolor}{RGB}{51, 51, 178} %引理定理环境颜色
\definecolor{bgmyupthmcolor}{RGB}{214, 214, 239}
\definecolor{bgmylowthmcolor}{RGB}{234, 234, 247}
\setbeamercolor{myupthmcolor}{fg=fgmyupthmcolor,bg=bgmyupthmcolor}
\setbeamercolor{mylowthmcolor}{fg=black,bg=bgmylowthmcolor}

\colorlet{temp}{blue!70!orange} %推论环境颜色
\colorlet{fgmyupcorocolor}{temp}
\colorlet{bgmyupcorocolor}{temp!20}
\colorlet{bgmylowcorocolor}{temp!10}
\setbeamercolor{myupcorocolor}{fg=fgmyupcorocolor,bg=bgmyupcorocolor}
\setbeamercolor{mylowcorocolor}{fg=black,bg=bgmylowcorocolor}

\colorlet{temp}{lime!60!black} %证明环境颜色
\colorlet{fgmyupproofcolor}{temp}
\colorlet{bgmyupproofcolor}{temp!20}
\colorlet{bgmylowproofcolor}{temp!10}
\setbeamercolor{myupproofcolor}{fg=fgmyupproofcolor,bg=bgmyupproofcolor}
\setbeamercolor{mylowproofcolor}{fg=black,bg=bgmylowproofcolor}

\definecolor{fgmyupexamcolor}{RGB}{0, 127, 0} %例子环境颜色
\definecolor{bgmyupexamcolor}{RGB}{204, 229, 204}
\definecolor{bgmylowexamcolor}{RGB}{229, 242, 229}
\setbeamercolor{myupexamcolor}{fg=fgmyupexamcolor,bg=bgmyupexamcolor}
\setbeamercolor{mylowexamcolor}{fg=black,bg=bgmylowexamcolor}

\colorlet{temp}{red} %警示环境颜色
\colorlet{fgmyupalertcolor}{temp}
\colorlet{bgmyupalertcolor}{temp!20}
\colorlet{bgmylowalertcolor}{temp!10}
\setbeamercolor{myupalertcolor}{fg=fgmyupalertcolor,bg=bgmyupalertcolor}
\setbeamercolor{mylowalertcolor}{fg=black,bg=bgmylowalertcolor}


%自定义定理类环境
\newenvironment{dfn}[1][]%定义环境
{
	\begin{beamerboxesrounded}[upper=myupdfncolor,lower=mylowdfncolor,shadow]{\dfnname{} \mycntthm{} \ifthenelse{\equal{#1}{}}{}{(#1)}}\ifthenelse{\equal{\thmseriesnumbering}{true}}{\addtocounter{mycntthm}{-1}\refstepcounter{mycntthm}}{}
}%
{
	\end{beamerboxesrounded}
}

\newenvironment{lemma}[1][]%引理环境
{
	\begin{beamerboxesrounded}[upper=myupthmcolor,lower=mylowthmcolor,shadow]{\lemmaname{} \mycntthm{} \ifthenelse{\equal{#1}{}}{}{(#1)}}\ifthenelse{\equal{\thmseriesnumbering}{true}}{\addtocounter{mycntthm}{-1}\refstepcounter{mycntthm}}{}
}%
{
	\end{beamerboxesrounded}
}

\newenvironment{thm}[1][]%定理环境
{
	\begin{beamerboxesrounded}[upper=myupthmcolor,lower=mylowthmcolor,shadow]{\thmname{} \mycntthm{} \ifthenelse{\equal{#1}{}}{}{(#1)}}\ifthenelse{\equal{\thmseriesnumbering}{true}}{\addtocounter{mycntthm}{-1}\refstepcounter{mycntthm}}{}
}%
{
	\end{beamerboxesrounded}
}

\newenvironment{coro}[1][]%推论环境
{
	\begin{beamerboxesrounded}[upper=myupcorocolor,lower=mylowcorocolor,shadow]{\coroname{} \mycntthm{} \ifthenelse{\equal{#1}{}}{}{(#1)}}\ifthenelse{\equal{\thmseriesnumbering}{true}}{\addtocounter{mycntthm}{-1}\refstepcounter{mycntthm}}{}
}%
{
	\end{beamerboxesrounded}
}

\renewenvironment{proof}[1][]%证明环境
{
	\begin{beamerboxesrounded}[upper=myupproofcolor,lower=mylowproofcolor,shadow]{\proofname}
}%
{
	\hfill$\blacksquare$
	\end{beamerboxesrounded}
}

\newenvironment{exam}[1][]%例环境
{
	\begin{beamerboxesrounded}[upper=myupexamcolor,lower=mylowexamcolor,shadow]{\examname{} \mycntexam{} \ifthenelse{\equal{#1}{}}{}{(#1)}}\ifthenelse{\equal{\thmseriesnumbering}{true}}{\addtocounter{mycntexam}{-1}\refstepcounter{mycntexam}}{}
}%
{
	\end{beamerboxesrounded}
}

\newenvironment{solu}%解答环境
{
	\begin{beamercolorbox}[rounded=true,shadow=true]{mylowexamcolor}
}%
{
	\end{beamercolorbox}
}

\renewenvironment{alert}[1][]%警示环境
{
	\begin{beamerboxesrounded}[upper=myupalertcolor,lower=mylowalertcolor,shadow]{\alertname}
}%
{
	\end{beamerboxesrounded}
}
